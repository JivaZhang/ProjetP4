\documentclass[10pt,a4paper]{article}
\usepackage[utf8]{inputenc}
\usepackage{amsmath}
\usepackage{amsfonts}
\usepackage{amssymb}
\usepackage{mathtools}

\newcommand{\PDeriv}[2]{\frac{\partial#1}{\partial#2}} % Pour faire des fractions derivees partielle sans trops se casser la tete :) (p. ex. \PDeriv{y(t}{t})
\newcommand{\V}[1]{\overrightarrow{#1}}
\newcommand{\Base}[1]{\widehat{\uline{a_{#1}}}} % Pour les vecteurs de base (MMC) par exemple \Base{1},\Base{2} et \Base{3}

\begin{document}
\title{Projet 4: MS1}
\date\today
\author{Groupe 7}
\maketitle

\section{Trilateration}
	La trilateration est une manière de localiser une cible à partir de trois récepteurs en se basant sur l'estimation du temps de vol entre la cible et un des récepteurs. 
		
		
	Afin de localiser la cible au moins trois récepteurs fixes sont nécessaire. En effet pour chaque récepteur, une fois que le temps de vol est connu, le lieu des points possible pour la position de la cible est un cercle. Entre deux récepteurs ce lieu devient au maximum deux points et avec trois celui-ci est au maximum un point unique. 
		\subsection{Résolution dans le cas idéal}
		Si chaque récepteur se trouve à la position $(x_i , y_i)$ et que sa distance par rapport à l'émetteur est $d_i$ , alors dans un cas idéal, la position de l'émetteur est solution du system \ref{sys}.  
		\begin{equation}
		\begin{dcases}
			(x - x_1)^2 + (y-y_1)^2 = d_1^2 \\
			(x - x_2)^2 + (y-y_2)^2 = d_2^2 \\
			(x - x_3)^2 + (y-y_3)^2 = d_3^2 
		\end{dcases}
		\label{sys}
		\end{equation}
		
		Pour trouver une solution à ce système il est possible de procéder de la manière suivante. Premièrement soustraire deux fois deux équations. Cette soustraction donne une équation de droite qui est donnée par l'équation \ref{eq1} si les équations $i$ et $j$ sont soustraites.
		
		\begin{equation}
			\label{eq1}
			y = \frac{(d_i^2 - d_j^2) - (y_i^2 -y_j^2)-(x_i^2 - x_j^2) + 2x(x_i-x_j)}{-2(y_i-y_j)}
		\end{equation}
		
		L'intersection de deux de ces droites est la solution du système \ref{sys}. 
		
		\subsection{Résolution dans le cas non-idéal}
			Dans le cas non-idéal le système \ref{sys} n'a pas de solution. Il est donc intéressant de chercher une solution approchée. Pour ce faire il faut trouver une expression de l'erreur commise et la minimiser. La distance d'un point $(x_p , y_p)$ à un cercle centrée en $(x_i , y_i)$ et de rayon $d_i$ ( $dist(P , C_i)$) est donnée par l'équation \ref{dist}. L'erreur commise pour une point $P$ est donc la somme de ces distances au carré (equation \ref{err}). 
			\begin{equation}
				dist(P , C_i)= |{\sqrt{(x_i-x_p)^2 +(y_i - y_p)^2} - d_i}| 
				\label{dist}
			\end{equation} 
			
			\begin{equation}
				\label{err}
				err(P) = \sum _{i=1} ^{3} (\sqrt{(x_i-x_p)^2 +(y_i - y_p)^2} - d_i)^2
			\end{equation}
			
			Afin de minimiser cette erreur, la fonction \texttt{Matlab fminsearch} a été utilisée. 
		\subsection{Calcul du temps de vol}
			
			Pour calculer le temps de vol l'opération de corrélation est utilisée. Celle-ci peut être effectuée dans le domaine fréquentiel (Equation \ref{corF}) ou dans le domaine temporel (Equation \ref{corT}). 
			\begin{equation}
				\label{corF}
				C(\omega) = R(\omega)S^*(\omega)
			\end{equation}
			\begin{equation}
				\label{corT}
				c(\tau) = r(t)*s(-t)
			\end{equation}
			
			Deux méthodes ont été implémentées. La première consiste à faire la corrélation sur tout le signal reçu en utilisant la fonction \texttt{xcorr} de \texttt{Matlab}.
			
			 
			La deuxième méthode consiste à diviser le signal en plusieurs parties. Dans le cas ici présent, un temps de vol est calculé pour chaque impulsion émises. Dans ce cas l'opération de corrélation se fait dans le domaine fréquentielle en utilisant les fonctions \texttt{fft} et \texttt{ifft} de \texttt{Matlab}. Le temps de vol utilisé est la moyenne de l'échantillon récolté sur chaque impulsion.
			
			\begin{figure}
				\centering
				\begin{tabular}{|l|l|l|}
					\hline
					fonction & temps moyen[s] & erreur moyenne[cm] \\
					\hline
					no-split & 0.8516 & 0.1711 \\
					split & 0.5857 & 0.2566 \\
					\hline
				\end{tabular}
				\caption{Comparaison des différentes méthodes}
			\end{figure}
			
						
\end{document}
