\documentclass[10pt,a4paper]{article}
\usepackage[utf8]{inputenc}
\usepackage{amsmath}
\usepackage{amsfonts}
\usepackage{amssymb}
\usepackage{url}
\usepackage{graphicx}

\begin{document}
\title{Projet 4: MS2}
\date\today
\author{Groupe 7}
\maketitle

\section{Extraction des répliques}
	\subsection*{Alignement des signaux}
	Afin d'extraire les répliques, les signaux reçus ($s(t)$) et émis ($r(t)$) doivent être aligner. Pour ce faire l'opération de corrélation est effectuée entre les deux signaux (voir Equation \ref{cor}). Le $\tau$ pour lequel la corrélation est maximum ($\hat{\tau}$). correspond au décalage temporel entre les deux signaux. 
	 
	\begin{equation}
	\label{cor}
	C(\tau) = \int s(t) * r(t-\tau) dt
	\end{equation}
	
	Afin de décaler le signal de $\hat{\tau}$ dans le domaine temporel, un changement de phase est opéré dans le domaine de Fourier. Pour chaque fréquence une opération est effectuée où $A(f)$ et $\theta(f)$ correspondent  à l'amplitude et à la phase de la fréquence $f$ du spectre du signal reçu. $Delaysig(f)$ est le spectre du signal décalé dans le temps. 
	
	Cette opération décale chaque fréquence d'une phase correspondant au décalage temporel voulu (voir Equation \ref{del}).  L'amplitude étant conservée pour chaque fréquence l'opération est donc celle de l'Equation \ref{op}. Il suffit alors de faire la transformée de Fourier inverse pour retrouver le signal décalé en temporel. 
	\begin{equation}
		\label{del}
			\theta(f)_{delay} = \theta(f) + (\hat {\tau} * f* 2\pi)
	\end{equation}
	
	\begin{equation}
		\label{op}
			Delaysig(f) = A(f)* e^{j*(\theta(f)_{delay})}
	\end{equation}
	
	\subsection*{Extraction du signal sans répliques}
			Les deux signaux $r(t)$ et $s(t)$ n'ont pas la même amplitude. Afin d'extraire les répliques, une opération linéaire est effectuée entre le $r(t-\hat{\tau})$ et $s(t)$ (voir Equation \ref{srt}).
			\begin{equation}
				\label{srt}
				s_r(t) = s(t) - Ar(t-\hat{\tau})
			\end{equation} 
			
			$A$ est trouvé au sens des moindres carrés. Cela signifie que le $A$ correspond (voir Equation \ref{car}). 
			
			\begin{equation}
				\label{car}
				min_A = \int s_r(t)^2 dt
			\end{equation} 
			
			Cette minimisation est effectuée à l'aide de la fonction \texttt{trapz()} de \texttt{MatLab} qui effective une intégration par la méthode des trapèzes et de la méthode d'optimisation de Newton-Raphson qui a été réimplémentée. 
\end{document}
